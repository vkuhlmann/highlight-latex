\documentclass{article}

\usepackage{amsmath}

\usepackage{latex-codemarkup}

\usepackage{geometry}
\usepackage{parskip}
\usepackage[dutch]{babel}
\usepackage{multicol}

\geometry{
	paperwidth=182mm,
	paperheight=80mm,
	margin=1.5cm
}

\cmdDefault{gekendcommando}
\cmdA{begin}
\cmdA{end}
\cmdB{inAccB}
\cmdC{inAccC}
\cmdD{inAccD}

\begin{document}
	\begin{multicols*}{2}
		Verander hiervoor de maximale nummerdiepte
		\begin{lstlisting}[belowskip=0pt,aboveskip=2pt]
% Here is some code
\setcounter{secnumdepth}{1}
\begin{document}
	\section{Met nummer (diepte 1)}
	
	\setcounter{secnumdepth}{0}
	\section{Zo ook hier geen nummers}
	\ongekendcommando\gekendcommando
	\inAccB\inAccC\inAccD
	\section ~\smash{\ensuremath{\sqrt{2}}}~
\end{document}
		\end{lstlisting}
		
		De \lstinline|\thesection| en \lstinline|\thesubsection| is hoe het nummer verschijnt. Het gedeelte \lstinline|\alph{subsection}| toont het ondertitel nummer in alfabetnummering.
		
		\lstset{classoffset=0,morekeywords={one,three,five},keywordstyle=\color{red},classoffset=1,morekeywords={two,four,six},keywordstyle=\color{blue},classoffset=0}
		\begin{lstlisting}
one two three four five six
		\end{lstlisting}
	\end{multicols*}	
\end{document}
