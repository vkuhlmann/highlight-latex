\documentclass{beamer}

\usepackage{../../highlight-latex}

\usetheme{Dresden}
\usecolortheme{dolphin}

\title{LaTeX highlight demo}
\author{Vincent Kuhlmann}
\date{13 March 2021}


% All options exposed here. Feel free to delete or comment out what you want!
% Colors accA ... accF. These are the default colors.

\colorlet{curlyBrackets}{red!50!blue}
\colorlet{squareBrackets}{blue!50!white}
\colorlet{codeBackground}{gray!10!white}

\updatehighlight{
	name = default,
	color = {blue!90!black},
	macros = {
		\gekendcommando
	}
}

\definecolor{accA}{RGB}{0,149,255}

\updatehighlight{
	name = accentA,
	color = accA,
	macros = {
		begin,end
	}
}

\colorlet{accB}{green!60!black}
\updatehighlight{
	name = accentB,
	color = accB,
	macros = {
		\inAccB
	},
	keywords = {
		nummer
	}
}


\colorlet{accC}{red!60!black}

\updatehighlight{
	name = accentC,
	color = accC,
	macros = {
		\inAccC	
	}
}

\colorlet{accD}{orange!100!black}

\updatehighlight{
	name = accentD,
	color = accD,
	macros = {
		\inAccD
	}
}

\begin{document}
	\begin{frame}
		\titlepage
		\centering
		Let's start
	\end{frame}

	\begin{savecode}{simpledocument}
		\begin{highlightblock}[linewidth=19em,gobble=4]
			% File: document.tex
			\documentclass[a4paper]{article}
			\usepackage{amsmath}
			
			\begin{document}
				Insert math like $\sqrt{2}$.
				
			\end{document}
		\end{highlightblock}
	\end{savecode}
	
	\begin{frame}
		\frametitle{Simple document}
		And your code should now look like:
		\copycode{simpledocument}
		after it
	\end{frame}
\end{document}
