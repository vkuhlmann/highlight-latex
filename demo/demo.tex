\documentclass{article}

\usepackage{amsmath}

% [unpublished] Relative path to the highlightlatex.sty. When it is in the same
% [unpublished] directory as this file, remove the '../'. Also do this if you
% [unpublished] have properly installed highlightlatex.
\usepackage[gobbletabs=3,tabsize=2]{../highlightlatex}

\usepackage{geometry}
\usepackage{parskip}
\usepackage[dutch]{babel}
\usepackage{multicol}
\usepackage{hyperref}

\geometry{
	paperwidth=182mm,
	paperheight=85mm,
	margin=1cm
}

\pagestyle{empty}

% Default config. These options can be set as package options as well.
\hllconfigure{
%	mathdollar = {
%		off
%	},
%	mathdollarcolor=red,
%	frame=lines,
%	tabsize=4,
%	gobble=0,
%  %gobbletabs=0,
%	backgroundcolor=gray!6!white,
%	bracecolor=red!50!blue,
%	bracketcolor=blue!50!white,
%	commentcolor=green!40!black,
%	alsoletter={$@_!|?$},
}

\updatehighlight{
	name = default,
	% 90% blue, remaining is black
	color = {blue!90!black},
	add = {
		\knowncommand
	},
	% You can't have blank lines when providing these key-values
	% (blank line means new paragraph, and that terminates this command)
	% So use percent signs (%) to comment out the newline character.
	%
	name = structure,
	add = {
		% Default:
		%\begin, \end
	},
	%^ This extra comma doesn't hurt. In fact, it helps when you're
	% reordering code; you don't want missing comma's.
}

\updatehighlight{
	% Some characters might not work as keyword. Don't go too
	% crazy.
	name = greenDollar,
	style = {\itshape\color{green!70!black}},
	add = {
		% The dollar sign is provided an extra time just to
		% calm down TeXstudio's code highlighting.
		$, $
	},
%
	name = accentA,
	color = green!60!black,
	add = {
		\inAccA, Hi!
	},
%
	name = accentB,
	color = red!60!black,
	add = {
		\inAccB
	},
%
	name = accentC,
	color = orange!100!black,
	add = {
		\inAccC
	}
}

\hllconfigure{
	mathdollar = {
		on,cumulative=false,color=red
	}
}

\begin{document}
	\begin{multicols}{2}
		And look at this beautiful code
		\begin{highlightblock}
			% Here is some code
			\setcounter{secnumdepth}{1}
			\begin{document}
				\section{My section (and Hi!)}
				
				\unknowncommand\knowncommand
				\inAccA\inAccB\inAccC
				\section ~\smash{\ensuremath{\sqrt{2}\;\leftarrow}} cool!~
				
				Insert literal tildes like ~\textasciitilde~. Hi!
			\end{document}
		\end{highlightblock}
		with some text after it.
		
		\vfill\leavevmode\columnbreak
		
		% The name allows us to modify what we had set for it.
		\updatehighlight{
			name = accentB,
			% Removes all commands and keywords from it
			clear,
			name = accentA,
			add = {
				\inAccB	
			}
		}
		
		Let's also show square brackets:
		\begin{highlightblock}
			% File: document.tex
			\documentclass[a4paper]{article}
			\usepackage{amsmath}
			
			\begin{document}
				Insert math like $\sqrt{2}$.
				\inAccA\inAccB\inAccC
				
			\end{document}
		\end{highlightblock}
	
		The first line was of the form \hll|\documentclass[]{}|. Very interesting.
		Note the \hll|$| are only green because we defined it as a keyword.
	\end{multicols}
	\begin{center}
		Go to \url{https://github.com/vkuhlmann/highlight-latex}
	\end{center}
\end{document}
