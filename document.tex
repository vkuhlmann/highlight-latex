\documentclass{article}

\usepackage{amsmath}

\usepackage{latex-codemarkup}

\usepackage{geometry}
\usepackage{parskip}
\usepackage[dutch]{babel}
\usepackage{multicol}

\geometry{
	paperwidth=182mm,
	paperheight=80mm,
	margin=1.5cm
}

% All options exposed here. Feel free to delete or comment out what you want!
% Colors accA ... accF. These are the default colors.
pa
\colorlet{accDefault}{blue!90!black}
\cmdDefault{gekendcommando}

\definecolor{accA}{RGB}{0,149,255}
\cmdA{begin}
\cmdA{end}

\colorlet{accB}{green!60!black}
\cmdB{inAccB}
\highlightB{nummers}

\colorlet{accC}{red!60!black}
\cmdC{inAccC}

\colorlet{accD}{orange!100!black}
\cmdD{inAccD}

\colorlet{accE}{green!60!black}

\colorlet{accF}{green!60!black}

\colorlet{codeBackground}{gray!10!white}

\colorlet{curlyBrackets}{red!50!blue}
\colorlet{squareBrackets}{blue!50!white}


\begin{document}
	\begin{multicols*}{2}
		Verander hiervoor de maximale nummerdiepte
		\begin{lstlisting}[belowskip=0pt,aboveskip=2pt]
% Here is some code
\setcounter{secnumdepth}{1}
\begin{document}
	\section{Met nummer (diepte 1)}
	
	\setcounter{secnumdepth}{0}
	\section{Zo ook hier geen nummers}
	\ongekendcommando\gekendcommando
	\inAccB\inAccC\inAccD
	\section ~\smash{\ensuremath{\sqrt{2}}}~
\end{document}
		\end{lstlisting}
		
		De \lstinline|\thesection| en \lstinline|\thesubsection| is hoe het nummer verschijnt. Het gedeelte \lstinline|\alph{subsection}| toont het ondertitel nummer in alfabetnummering.
	
		\lstset{classoffset=0,morekeywords={one,three,five},keywordstyle=\color{red},classoffset=1,morekeywords={two,four,six},keywordstyle=\color{blue},classoffset=0}
		\begin{lstlisting}
one two three four five six
		\end{lstlisting}
	\end{multicols*}	
\end{document}
