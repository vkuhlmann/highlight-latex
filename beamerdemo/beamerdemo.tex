\documentclass{beamer}

% [unpublished] Relative path to the highlightlatex.sty. When it is in the same
% [unpublished] directory as this file, remove the '../'. Also do this if you
% [unpublished] have properly installed highlightlatex.
\usepackage[debugframe]{../highlightlatex}

\usetheme{Dresden}
\usecolortheme{dolphin}
\useoutertheme{miniframes}

\usepackage{hyperref}

\title{LaTeX highlight demo}
\author{Vincent Kuhlmann}
\date{14 March 2021}

% Override if you want. Else you can delete it.
%\colorlet{curlyBrackets}{red!50!blue}
%\colorlet{squareBrackets}{blue!50!white}
%\colorlet{codeBackground}{gray!10!white}
%\colorlet{comment}{green!40!black}
\def\defaultgobble{6}

\updatehighlight{
	name = default,
	% 90% blue, remaining is black
	color = {blue!90!black},
	add = {
		\knowncommand
	},
	% You can't have blank lines when providing these key-values
	% (blank line means new paragraph, and that terminates this command)
	% So use percent signs (%) to comment out the newline character.
	%
	name = structure,
	add = {
		% Default:
		%\begin, \end
	},
	%^ This extra comma doesn't hurt. In fact, it helps when you're
	% reordering code; you don't want missing comma's.
}

\updatehighlight{
	% Some characters might not work as keyword. Don't go too
	% crazy.
	name = greenDollar,
	style = {\itshape\color{green!70!black}},
	add = {
		% The dollar sign is provided an extra time just to
		% calm down TeXstudio's code highlighting.
		$, $
	},
	%
	name = accentA,
	color = green!60!black,
	add = {
		\inAccA, Hi!
	},
	%
	name = accentB,
	color = red!60!black,
	add = {
		\inAccB
	},
	%
	name = accentC,
	color = orange!100!black,
	add = {
		\inAccC
	}
}

% https://tex.stackexchange.com/questions/8260/what-are-the-various-units-ex-em-in-pt-bp-dd-pc-expressed-in-mm
\makeatletter
%http://groups.google.com/group/comp.text.tex/msg/7e812e5d6e67fcc5
\def\convertto#1#2{\strip@pt\dimexpr #2*65536/\number\dimexpr 1#1}
\makeatother

\newlength\mystdim

\begin{document}
	\section{Beautiful code}
	\begin{frame}
		\titlepage
		\centering
		Let's start
	\end{frame}

	\begin{saveblock}{arealjumble}
%		\the\ht\strutbox, \the\baselineskip, \the\dimexpr .7\baselineskip\relax, \the\dimexpr .3\baselineskip\relax
		\begin{highlightblock}[linewidth=19em,gobble=6,framerule=5.251792907714844pt]
			% Here is some code
			\setcounter{secnumdepth}{1}
			\begin{document}
				\section{My section (and Hi!)}
				
				\unknowncommand\knowncommand
				\inAccA\inAccB\inAccC
				\section ~\smash{\ensuremath{\sqrt{2}\;\leftarrow}} cool!~
				
				Insert literal tildes like ~\textasciitilde~. Hi!
			\end{document}
		\end{highlightblock}
	\end{saveblock}
		
	\begin{frame}
		\frametitle{Beautiful code}
		And look at this beautiful code
		\useblock{arealjumble}
		%with some text after it.
		
		%\the\ht\csname @codebox@arealjumble\endcsname, \the\dp\csname @codebox@arealjumble\endcsname,
		%\the\wd\csname @codebox@arealjumble\endcsname
		
		%\the\ht\strutbox, \the\baselineskip, \the\dimexpr .7\baselineskip\relax, \the\dimexpr .3\baselineskip\relax
		
	\end{frame}

	\begin{frame}
		\frametitle{Beautiful code}
		Even more beautiful when we center it:
		\begin{center}
			\consumeblock{arealjumble}
		\end{center}
	
		\setlength\mystdim{5.251792907714844pt}
		
		pt: \convertto{pt}{\mystdim}, mm: \convertto{mm}{\mystdim},
		ex: \convertto{ex}{\mystdim}, pc: \convertto{pc}{\mystdim},
		in: \convertto{in}{\mystdim}, baselinefrac \convertto{\baselineskip}{\mystdim},
		pageheightfrac \convertto{\paperheight}{\mystdim}
	\end{frame}

	% The name allows us to modify what we had set for it.
	\updatehighlight{
		name = accentB,
		% Removes all commands and keywords from it
		clear,
		name = accentA,
		add = {
			\inAccB	
		}
	}
	
	\begin{saveblock}{nextpart}
		\begin{highlightblock}[linewidth=0.6\textwidth,framerule=3pt]
			% File: document.tex
			\documentclass[a4paper]{article}
			\usepackage{amsmath}
			
			\begin{document}
				Insert math like $\sqrt{2}$.
				\inAccA\inAccB\inAccC
				
			\end{document}
		\end{highlightblock}
	\end{saveblock}

	\begin{frame}
		\frametitle{Continuing}
		Let's also show square brackets:
		
		\begin{center}
			\useblock{nextpart}
		\end{center}
		
		\pause
		% Note we need to escape characters!
		The first line was of the form \hll|\\documentclass[]\{\}|. Very interesting.
		
		\pause
		Note the \hll|\$| are only green because we defined it as a keyword.
	\end{frame}

	\begin{saveblock}{basicfigure}
		\begin{highlightblock}[linewidth=0.6\textwidth]
			\begin{figure}
				\includegraphics
				[width=0.9\linewidth]
				{myPlot.pdf}
				
				\caption{My plot}
				\label{fig:myplot}
			\end{figure}
		\end{highlightblock}
	\end{saveblock}

	\begin{saveblock}{figureplacement}
		\begin{highlightblock}[linewidth=0.4\textwidth]
			\begin{figure}[htbp]
		\end{highlightblock}
	\end{saveblock}

	\begin{frame}
		\frametitle{Next to eachother}
		\begin{columns}
			\begin{column}{0.6\textwidth}
				\consumeblock{basicfigure}
			\end{column}
			\begin{column}{0.4\textwidth}
				You could have here code too:
				\consumeblock{figureplacement}
				
			\end{column}
		\end{columns}
	\end{frame}

	\section{Repository}
	\begin{frame}
		\begin{center}
			{\Large Go to}\\
			\url{https://github.com/vkuhlmann/highlight-latex}
		\end{center}
	\end{frame}
	
\end{document}
